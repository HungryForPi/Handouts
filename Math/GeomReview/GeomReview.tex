\documentclass{scrartcl}
\usepackage[usenames,dvipsnames,svgnames]{xcolor}
\usepackage[shortlabels]{enumitem}
\usepackage[framemethod=TikZ]{mdframed}
\usepackage{amsmath,amssymb,amsthm}
\usepackage{epigraph}
\usepackage[colorlinks]{hyperref}
\usepackage{microtype}
\usepackage{mathtools}
\usepackage[headsepline]{scrlayer-scrpage}
\usepackage{thmtools}
\renewcommand{\epigraphsize}{\scriptsize}
\renewcommand{\epigraphwidth}{60ex}
\addtolength{\textheight}{3.14cm}
\ihead{\footnotesize\textbf{Geometry Review}}
\ohead{\footnotesize \today}
\providecommand{\clubs}[1]{$#1\clubsuit$}
\providecommand{\ol}{\overline}
\providecommand{\eps}{\varepsilon}
\providecommand{\half}{\frac{1}{2}}
\providecommand{\dang}{\measuredangle} %% Directed angle
\providecommand{\CC}{\mathbb C}
\providecommand{\FF}{\mathbb F}
\providecommand{\NN}{\mathbb N}
\providecommand{\QQ}{\mathbb Q}
\providecommand{\RR}{\mathbb R}
\providecommand{\ZZ}{\mathbb Z}
\providecommand{\dg}{^\circ}
\providecommand{\ii}{\item}
\providecommand{\alert}[1]{{\sffamily\textbf{\textcolor{blue}{#1}}}}

\reversemarginpar


\mdfdefinestyle{mdbluebox}{roundcorner=10pt,innerbottommargin=9pt,
linecolor=blue,backgroundcolor=TealBlue!5,}
\declaretheoremstyle[headfont=\sffamily\bfseries\color{MidnightBlue},
mdframed={style=mdbluebox},]{thmbluebox}

\mdfdefinestyle{mdredbox}{frametitlefont=\bfseries,innerbottommargin=8pt,
nobreak=true,backgroundcolor=Salmon!5,linecolor=RawSienna,}
\declaretheoremstyle[headfont=\bfseries\color{RawSienna},
mdframed={style=mdredbox},headpunct={\\[3pt]},postheadspace=0pt,]{thmredbox}

\mdfdefinestyle{mdgreenbox}{linecolor=ForestGreen,backgroundcolor=ForestGreen!5,
linewidth=2pt,rightline=false,leftline=true,topline=false,bottomline=false,}
\declaretheoremstyle[headfont=\bfseries\sffamily\color{ForestGreen!70!black},
mdframed={style=mdgreenbox},headpunct={ --- },]{thmgreenbox}

\mdfdefinestyle{mdorangebox}{linecolor=RedOrange,backgroundcolor=RedOrange!5,
linewidth=2pt,rightline=false,leftline=true,topline=false,bottomline=false,}
\declaretheoremstyle[headfont=\bfseries\sffamily\color{RedOrange!70!black},
mdframed={style=mdorangebox},headpunct={ --- },]{thmorangebox}

\mdfdefinestyle{mdblackbox}{linecolor=black,backgroundcolor=RedViolet!5!gray!5,
linewidth=3pt,rightline=false,leftline=true,topline=false,bottomline=false,}
\declaretheoremstyle[mdframed={style=mdblackbox}]{thmblackbox}

\declaretheorem[style=thmredbox,name=Problem,numberwithin=subsection]{problem}
\declaretheorem[style=thmredbox,name=Example,numberwithin=subsection]{example}
\declaretheorem[style=thmredbox,name=$\bigstar$ Problem,sibling=problem]{niceprob}

\declaretheorem[style=thmbluebox,name=Theorem,sibling=example]{theorem}
\declaretheorem[style=thmbluebox,name=Proposition,sibling=example]{prop}
\declaretheorem[style=thmbluebox,name=Corollary,sibling=theorem]{corollary}
\declaretheorem[style=thmbluebox,name=Lemma,sibling=theorem]{lemma}
\declaretheorem[style=thmbluebox,name=Theorem,numbered=no]{theorem*}
\declaretheorem[style=thmbluebox,name=Lemma,numbered=no]{lemma*}

\declaretheorem[style=thmgreenbox,name=Claim,numberwithin=theorem]{claim}
\declaretheorem[style=thmgreenbox,name=Claim,numbered=no]{claim*}
\declaretheorem[style=thmgreenbox,name=Definition,sibling=theorem]{definition}
\declaretheorem[style=thmgreenbox,name=Definition,numbered=no]{definition*}

\declaretheorem[style=thmblackbox,name=Remark,sibling=theorem]{remark}
\declaretheorem[style=thmblackbox,name=Remark,numbered=no]{remark*}

\declaretheorem[style=thmorangebox,name=Warning,sibling=theorem]{warning}
\declaretheorem[style=thmorangebox,name=Note,numbered=no]{note}
\declaretheorem[style=thmblackbox,name=Exercise,numbered=no]{exercise}
%% 426c616e6b204c615465587e

\usepackage{graphicx}
\usepackage{todonotes}

\newenvironment{soln}{\begin{proof}[Solution]}{\end{proof}}
\newenvironment{walkthrough}{\noindent\textbf{Walkthrough.}}{}
\newlist{walk}{enumerate}{3}
\setlist[walk]{label=\bfseries (\alph*)}
\usepackage{todonotes}

\providecommand{\pow}{\operatorname{Pow}}
\providecommand{\vocab}[1]{{\textbf{\textcolor{ForestGreen}{#1}}}}

\title{Geometry Review}
\subtitle{Stuyvesant Senior Math Team}
\author{Aditya Pahuja}
\date{\today}
\begin{document}
\maketitle
\tableofcontents
\pagebreak

\setcounter{section}{-1}
\section{Conventions}
For the most part, diagrams will not be included,
both because I'm lazy and because you should get practice drawing diagrams!
Drawing a decent diagram can often be the hardest part of a problem.

\subsection{Definitions}
It's assumed that you know the definitions of the five
most common triangle centers, which are listed here for completeness.
In $\triangle ABC$,
\begin{itemize}
	\ii the \vocab{incenter} is the concurrency point of the
	(internal) angle bisectors of $\angle A$, $\angle B$, and $\angle C$.
	\ii the \vocab{centroid} is the concurrency point of the medians.
	\ii the \vocab{circumcenter} is the concurrency point of the perpendicular
	bisectors of $AB$, $BC$, and $CA$.
	\ii the \vocab{orthocenter} is the concurrency point of the altitudes.
	\ii the \vocab{$A$-excenter} is the concurrency point of
	the internal angle bisector of $\angle A$ and
	the external angle bisectors of $\angle B$ and $\angle C$.
	The $B$-excenter and $C$-excenter are defined similarly.
\end{itemize}
The existence of these centers will be assumed for now;
methods to prove them will be developed later on.
(You should be able to prove the all of the centers' existence
if you remember last year's lessons, though.)
\subsection{Notation}
\begin{itemize}
	\ii When writing $\triangle ABC\cong\triangle XYZ$ or
	$\triangle ABC\sim\triangle XYZ$, the order of the vertices encodes
	corresponding vertices. In other words,
	$\angle A = \angle X$, $\angle B = \angle Y$, and $\angle C = \angle Z$.
	\ii If points $P_1$, $P_2$, \dots, $P_n$ are concyclic
	(i.e. lie on some circle), then
	$(P_1P_2\dots P_n)$ is the circle passing through those points.
	\ii We sometimes denote the side lengths of the sides of $\triangle ABC$ by
	$BC = a$, $CA = b$, and $AB = c$.
\end{itemize}
\pagebreak
\section{Philosophical Rambling}
There are two main ways to view a geometry problem:
\emph{synthetically} and \emph{analytically}.

Analytic methods involve viewing the configuration using
some computational framework: trigonometry and Cartesian coordinates
are the standard beginner's tools for this approach.

Synthetic geometry, on the other hand, can be thought of as building up
geometry from some set of axioms: A lot of your 9th-grade geometry
is the basis of synthetic methods.

A strong geometer is comfortable with both synthetic and analytic techniques:
each tool (or perhaps combination of tools) is best suited for different
kinds of problems. In my experience, people are generally much more comfortable
working with computational techniques, so I'll generally focus more on
synthetic techniques here. As such, you should always try to look for
synthetic solutions where you can
(but bonus points if you can find multiple solutions!).

\section{Synthetic Techniques}
\subsection{Digression: How to approach problems}
Generally, solving problems comes in two phases:
\begin{enumerate}
	\ii The ``scouting'' phase, where you try to get some intuition
	about why the problem works.
	This can manifest in multiple ways, such as
	\begin{itemize}
		\ii searching for what ``should'' be true, such as by working backwards
		(ex. ``$\triangle HBC\sim\triangle ODE$ needs to be true
		for the problem to be true.''),
		\ii getting a heuristic understanding of what's going on
		(ex. ``$|7^a - 3^b|$ should generally be much larger than $|a^4 + b^2|$,
		so $a$ and $b$ should be small, whatever \emph{small} means.''),
		\ii thinking about why certain techniques don't work
		(ex. ``I can't show that a randomly selected path in my graph
		behaves the way I want, so I should try considering
		the whole set of paths simultaneously.''),
		\ii thinking about what techniques might work
		(ex. ``I have a central right triangle, and my points are easy to define
		with respect to this right triangle, so Cartesian coordinates could work.'')
	\end{itemize}
	and so on. These are the things that people sometimes call
	``motivation.''
	\ii The ``attacking'' phase, where you prove things about the problem.
	This is the part where you actually try computing things, performing induction,
	etc., ideally solving the problem, but at least getting some sort of
	intermediate claim. These are much more concrete methods,
	and are the parts that actually show up in your solution,
	if you were to write it out.
\end{enumerate}
(This is largely parroting ideas from Evan Chen's blog post on
\hyperlink{https://blog.evanchen.cc/2019/05/03/hard-and-soft-techniques/}{hard and soft techniques}.)

Put more simply, you gather information while scouting,
and then use that information to mount an attack, hopefully destroying the problem.\\

Most of the time, scouting is done through a synthetic lens.
This can come in the form of redefining points to be more well-behaved
(perhaps in hope of finding an analytic approach),
angle chasing to look for similar triangles/cyclic quadrilaterals
(although that can be thought of as attacking, too), etc.
In that vein, scouting strategies will be in \alert{blue, sans-serif font}.
Try to come up with some heuristics on your own, too.

\subsection{Angles}
Angle chasing is a very low-powered technique:
you should've already encountered most of the necessary material
in 9th-grade geometry.
For completeness, here are the two most important theorems that
you should know at least as well as the back of your hand
(but preferably better).

\begin{theorem}[Cyclic quadrilaterals and inscribed angles]
	Let $ABCD$ be a convex quadrilateral. Then, the following are equivalent:
	\begin{itemize}
		\ii There exists a circle containing $A$, $B$, $C$, and $D$.
		\ii $\angle ABC = 180\dg - \angle ADC$.
		\ii $\angle ACB = \angle ADB$.
	\end{itemize}
\end{theorem}

\begin{theorem}[Tangents and chords]
	There is a triangle $\triangle ABC$ and a line $\ell$ through $A$.
	Let $D$ be a point on $\ell$ such that $B$ and $D$ are on opposite sides
	of line $AC$.
	Then, $\ell$ is tangent to $(ABC)$ at $A$ if and only if
	$\angle ABC$ and $\angle DAC$ are congruent.
\end{theorem}

\subsubsection{Walkthroughs}
\begin{example}[JMO 2011/5]
	Points $A$, $B$, $C$, $D$, $E$ lie on a circle $\omega$
	and $P$ lies outside the circle. The given points are such that
	(i) lines $PB$ and $PD$ are tangent to $\omega$,
	(ii) $P$, $A$, $C$ are collinear, and
	(iii) $DE\parallel AC$.
	
	Prove that $BE$ bisects $AC$.
\end{example}

\begin{walkthrough}
	Let $M$ be the midpoint of $AC$. We can then phrase the problem as
	``Show that $B$, $M$, and $E$ are collinear.''
	\begin{walk}
		\ii Assume for a moment that $B$, $M$, and $E$ are collinear.
		What quadruple of points (besides a subset of $\{A, B, C, D, E\})$
		must form a cyclic quadrilateral under this assumption?
		
		\ii Get rid of the assumption and prove that the quadrilateral
		you found in \textbf{(a)} really is cyclic.
		
		\ii Using this cyclic quadrilateral, show that $B$, $M$, and $E$
		are collinear. One approach is to show that $\angle BMC = \angle BED$
		(assuming that $A$, $B$, $C$, $D$, $E$ are on the circle in that order).
	\end{walk}
	
	The key step of this problem is the discovery of the cyclic quadrilateral.
	Often, the key idea in a geometry problem is locating a particular
	cyclic quadrilateral, set of collinear points, set of concurrent lines,
	etc., so it's important to develop some strategies for guessing
	when these features appear.
	
	One such strategy, which we used in part \textbf{(a)}, is
	\alert{working backwards}. The idea is to assume that the desired result
	is true, and then formulate equivalent but more tractable statements.
	In this case, $M$ turns out to be the midpoint \emph{if and only if}
	this mystery quadrilateral is cyclic, so proving that it's cyclic
	essentially solves the problem.
	
	(Question: Why should you expect the quadrilateral being cyclic to be
	\emph{equivalent} to the collinearity?)
\end{walkthrough}

\begin{example}[2022 AIME II/11]
	Let $ABCD$ be a convex quadrilateral with $AB = 2$, $AD = 7$, and $CD = 3$
	such that the bisectors of acute angles $\angle DAB$ and $\angle ADC$
	intersect at the midpoint of $BC$. Find the area of $ABCD$.
\end{example}
\begin{walkthrough}
	The core of this problem is a bunch of angle chasing, followed by
	a tiny bit of computation.
	
	Let $M$ be the midpoint of $BC$ and let $P$ be the intersection of
	lines $AB$ and $CD$; then, $M$ is the incenter of $\triangle APD$.
	\begin{walk}
		\ii Solve the following problem first: Let $\triangle ABC$ be
		a triangle with incenter $I$. Points $X$ and $Y$ lie on $AB$ and $AC$
		such that $I$ is the midpoint of $XY$. Describe a ruler-and-compass
		construction for $X$ and $Y$ given $A$, $B$, $C$, and $I$.
		\ii How does the previous part apply in the original problem?
		Describe $\triangle PBC$.
		\ii Show that
		$\triangle ABM\sim\triangle MCD\sim\triangle AMD$.
		\ii Compute the lengths $AM$, $BM$, $CM$, and $DM$
		via the similar triangles.
		\ii Extract the final answer of $6\sqrt5$, either by Heron's formula
		or by trigonometry.
	\end{walk}
	This is a problem where the original diagram is hard to construct
	with a ruler and compass without some additional thinking.
	In such problems, one common approach is to
	\alert{try phrasing the problem more naturally}:
	in this case, rewriting the problem with respect to
	$\triangle APD$ essentially solved the problem, since it allowed us
	to wrangle the ``angle bisectors concur on the midpoint of $BC$'' condition.
\end{walkthrough}

\begin{example}[2020 AIME II/15]
	Let $\triangle ABC$ be an acute scalene triangle with circumcircle $\omega$.
	The tangents to $\omega$ at $B$ and $C$ intersect at $T$.
	Let $X$ and $Y$ be the projections of $T$ onto lines $AB$ and $AC$,
	respectively.
	Suppose $BT = CT = 16$, $BC = 22$, and $TX^2 + TY^2 + XY^2 = 1143$.
	Find $XY$.
\end{example}

\begin{walkthrough}
	There are several viable approaches here, including trigonometry
	and complex numbers. We will go through a synthetic solution, though.
	
	The most difficult aspect of this problem is that the diagram is very bare:
	angle chasing will show you a few equal angles, but you won't find
	anything particularly substantive.
	
	\begin{walk}
		\ii Let $M$ be the foot of the perpendicular from $T$ to $BC$.
		What properties does $M$ have?
		(There are at least three nontrivial ones.)
		\ii Show that $TXMY$ is a parallelogram.
		\ii The parallelogram law (also called Apollonius' theorem)
		says that
		\[TX^2 + TY^2 + MX^2 + MY^2 = XY^2 + TM^2.\]
		In other words, the sum of the squares of a parallelogram's side lengths
		is equal to the sum of the squares of its diagonals.
		Use this to compute the final answer.
		\ii Optionally, prove that $M$ is the orthocenter of $\triangle AXY$.
	\end{walk}
	
	Actually, this solution is pretty short: there are not many things to do
	once you add in $M$; the challenge is realizing that adding $M$ helps.
\end{walkthrough}

\subsubsection{Extra Problems}
In general, these will be problems that show some common configurations
or that I simply think are nice/instructive.
A couple of problems are marked with a star
because I think they're extra cool.
They are roughly in difficulty order.
\begin{problem}
	Let $\triangle ABC$ have orthocenter $H$. Lines $AH$, $BH$, and $CH$
	intersect lines $BC$, $CA$, and $AB$ at $D$, $E$, and $F$ respectively.
	Find all six cyclic quadrilaterals with vertices in
	$\{A, B, C, D, E, F, H\}$, and describe their circumcenters.
	
	(Bonus: prove that the six circumcenters are concyclic, too.)
\end{problem}

\begin{problem}[Incenter-Excenter Lemma]
	Consider a triangle $\triangle ABC$. The angle bisector of
	$\angle BAC$ intersects its circumcircle $\omega$ again at $L$.
	Show that $L$ is the circumcenter of quadrilateral
	$BICI_A$, where $I$ is the incenter and $I_A$ is the $A$-excenter
	of the triangle.
\end{problem}

\begin{problem}[Reflecting the orthocenter]
	Show that the reflections of the orthocenter of triangle $\triangle ABC$
	over $BC$ and the midpoint of $BC$ both lie on the circumcircle
	of $\triangle ABC$. Moreover, prove that the reflection over the midpoint
	is the point diametrically opposite from $A$.
\end{problem}

\begin{problem}[2021 AIME II/14]
	Let $\triangle ABC$ be an acute triangle with circumcenter $O$ and
	centroid $G$. Let $X$ be the intersection of the line tangent
	to the circumcircle of $\triangle ABC$ at $A$ and the line
	perpendicular to $GO$ at $G$.
	Let $Y$ be the intersection of lines $XG$ and $BC$.
	Given that the measures of $\angle ABC, \angle BCA,$ and $\angle XOY$
	are in the ratio $13 : 2 : 17$, compute the degree measure of $\angle BAC$. 
\end{problem}

\begin{niceprob}[USAMO 2021/1]
	Rectangles $BCC_1B_2$, $CAA_1C_2$, and $ABB_1A_2$ are erected
	outside an acute triangle $ABC$. Suppose that
	\[\angle BC_1C + \angle CA_1A + \angle AB_1B = 180\dg.\]
	Prove that lines $B_1C_2$, $C_1A_2$, and $A_1B_2$ are concurrent.
\end{niceprob}

\begin{niceprob}[BrMO 2013/2, aka the first isogonality lemma]
	Let $ABC$ be a triangle and let $P$ be a point inside it satisfying
	$\angle ABP = \angle PCA$. Let $Q$ be the reflection of $P$
	across the midpoint of $BC$. Prove that $\angle BAP = \angle CAQ$.
\end{niceprob}

\pagebreak
\subsection{Power of a Point}
Power of a point is an extremely useful theorem,
and I dedicate a section to it here because it's the most frequently used
bridge between length information and angle information
(except for maybe trigonometry).
This should make sense, as it's essentially similar triangles in a neat package.

Here's the statement as a refresher:
\begin{theorem}[Power of a point]
	Let $\Gamma$ be a circle and $P$ be any point. Then, across all choices of
	lines through $P$ intersecting $\Gamma$ at $X$ and $Y$
	(it's possible for $X = Y$), the quantity
	\[PX\cdot PY\]
	is constant.
\end{theorem}

It's also often phrased in the following equivalent manner:
\begin{corollary}
	Let $A$, $B$, $C$, and $D$ be points on a circle. Then, if $AB$ and $CD$
	intersect at a point $P$,
	\[PA\cdot PB = PC\cdot PD.\]
\end{corollary}
As alluded to above, the proof of this follows directly from
$\triangle PAC\sim\triangle PDB$.

Also, the converse is true!
\begin{theorem}[Converse of POP]
	If $AB$ and $CD$ are lines intersecting at $P$ and
	\[PA\cdot PB = PC\cdot PD,\]
	then $A$, $B$, $C$, and $D$ form a cyclic quadrilateral.
\end{theorem}

The nice thing about everything above is that we have a lot of freedom:
we can choose $AB$ and $CD$ to be any lines passing through $P$.
In particular this quantity is entirely dependent on the choice of
$P$ and the features of the circle. This motivates the following definition:

\begin{definition}
	Let $\Gamma$ be a circle with center $O$ and radius $r$,
	and let $P$ be a point in the plane.
	We define the \vocab{power} of $P$ with respect to $\Gamma$ by
	\[\pow_{\Gamma}(P) = OP^2 - r^2.\]
\end{definition}
\begin{exercise}
	Verify that $|\pow_\Gamma(P)| = PX\cdot PY$ for any choice of
	$X$ and $Y$ on $\Gamma$ such that $P$, $X$, and $Y$ are collinear.
	When is the power positive? Negative? Zero?
\end{exercise}
\pagebreak
While we will discuss this function in more detail later in the year,
here's one important result that shows up pretty often.

\begin{theorem}[Radical axis]
	Let $\Gamma_1$ and $\Gamma_2$ be two non-concentric circles.
	Then, the set of points $P$ satisfying
	\[\pow_{\Gamma_1}(P) = \pow_{\Gamma_2}(P)\]
	is a line that is perpendicular to the line joining the centers of
	$\Gamma_1$ and $\Gamma_2$. This line is called the
	\vocab{radical axis} of the two circles.
\end{theorem}

\begin{proof}
	Let $\Gamma_i$ have center $O_i$ and radius $r_i$. Then, we want to find
	the set of points $P$ for which
	\[O_1P^2 + r_1^2 = O_2P^2 - r_2^2 \iff
	O_1P^2 - O_2P^2 = r_1^2 - r_2^2.\]
	We then want to show that the set of points for which
	$O_1P^2 - O_2P^2$ is equal to the constant $r_1^2 - r_2^2$
	is a line.
	
	\begin{claim}
		Let $AB$ be a segment and let $C$ and $D$ be two points.
		Then,
		\[AC^2 - BC^2 = AD^2 - BD^2\]
		if and only if $AB\perp CD$.
	\end{claim}
	\begin{proof}
		If $AB\perp CD$, then the result follows from Pythagorean theorem.
		
		In the other direction, if the length condition is true, then
		let $H_C$ and $H_D$ be the feet of $C$ and $D$ onto $AB$. Then
		\[AH_C^2 - BH_C^2 = AC^2 - BC^2 = AD^2 - BD^2 = AH_D^2 - BH_D^2\]
		implies that $H_C = H_D$, so $CD\perp AB$.
	\end{proof}
	
	This says that the set of points $X$ for which
	$O_1X^2 - O_2X^2$ is constant is a line perpendicular to $O_1O_2$,
	which is what we wanted to prove.
\end{proof}

An alternate (and simpler) proof can also be extracted using coordinates,
but the synthetic approach is good to know, since
the intermediary claim is pretty useful itself.

\begin{remark}
	You may notice the function $f\colon\RR^2\to\RR$
	given by $\pow_{\omega_1}(X)-\pow_{\omega_2}(X)$
	arising naturally from the proof --- the radical axis
	is the set of points which make the function zero.
	Interestingly, $f$ is \vocab{linear}; that is,
	\[f(kA + (1-k)B) = kf(A) + (1-k)f(B)\]
	where multiplication (between real numbers and points) and
	addition (between two points) are done componentwise with the coordinates.
	
	The upshot of this is that knowing the value of $f$ at $A$ and $B$
	lets you compute $f(X)$ for all $X$ on line $AB$,
	and, if you know $f$ at $A$, $B$, and $C$, then you can compute
	the value of $f$ at any point in the plane.
\end{remark}

\pagebreak
\subsubsection{Walkthroughs}
\begin{example}[JMO 2012/1]
	Given a triangle $ABC$, let $P$ and $Q$ be on segments $AB$ and $AC$,
	respectively, such that $AP = AQ$. Let $S$ and $R$ be distinct points
	on segment $BC$ such that $S$ lies between $B$ and $R$,
	$\angle BPS = \angle PRS$, and $\angle CQR = \angle QSR$.
	Prove that $P$, $Q$, $R$, $S$ are concyclic.
\end{example}

\begin{walkthrough}
	This problem is one of my favorites because its solution
	uses a really unique idea.
	\begin{walk}
		\ii Assume for the sake of contradiction that $(PRS)$ and $(QRS)$
		are distinct circles. Then, show that
		$AB$ is tangent to $(PRS)$ at $P$ and
		$AC$ is tangent to $(QRS)$ at $Q$.
		\ii Convince yourself that $RS$ is the radical axis of the circles
		(in particular, that the two circles are not concentric).
		\ii Find a point not on line $RS$ that has the same power
		with respect to $(PRS)$ and $(QRS)$, and deduce a contradiction.
	\end{walk}
	
	I think this solution seems really unnatural at first glance,
	but it presents itself after noticing that the circles above
	should be tangent to $AB$ and $AC$.
	(Theorem 2.2 paying back in spades!)
\end{walkthrough}

\begin{example}[Euler's theorem]
	Let $\triangle ABC$ have circumcenter $O$ and incenter $I$,
	as well as circumradius $R$ and inradius $r$. Then,
	\[OI^2 = R^2 - 2Rr.\]
\end{example}

\begin{walkthrough}
	This problem lends itself very handily to power of a point
	because of the given equation.
	\begin{enumerate}[label=\textbf{(\alph*)}]
		\ii Let $\Gamma$ be the circumcircle of $\triangle ABC$.
		Show that $\pow_{\Gamma}(I) = -2Rr$.
		This means we want to show that $XI\cdot YI = 2Rr$
		for some choice of $X,Y\in\Gamma$ with $X$, $I$, $Y$ collinear.
		\ii Draw the angle bisector from $A$, and let it intersect $\Gamma$
		at $D$. Then, we want $AI\cdot ID = 2Rr$.
		If you haven't already done Problem 1.2, show that $DI = DB$.
		\ii Imagine that the equation above was the result of
		setting up a proportion between two similar triangles.
		Working backwards, what might the original proportion be?
		Keep in mind which segments would most readily fit in a triangle together.
		\ii If you did part \textbf{(c)} correctly, then you should be able to
		draw in one more point to create a pair of similar triangles
		with the desired proportion.
	\end{enumerate}
	
	This is a pretty classical example of power of a point, and it's
	sometimes used in geometric inequalities, since you can show that
	\[R(R - 2r) = OI^2 \ge 0,\]
	so $R \ge 2r$ (when does equality happen?). Using some area formulas,
	you can even extract the inequality
	\[4R^2 \ge \frac{abc}{a + b + c},\]
	which has no trace of $r$, surprisingly.
\end{walkthrough}

\subsubsection{Extra Problems}

\begin{problem}
	Let $\Gamma_1$ and $\Gamma_2$ be two intersecting circles.
	Let a common tangent to $\Gamma_1$ and $\Gamma_2$ touch
	$\Gamma_1$ at $A$ and $\Gamma_2$ at $B$.
	Show that the common chord of $\Gamma_1$ and $\Gamma_2$, when extended,
	bisects segment $AB$.
\end{problem}

\begin{problem}[Radical center]
	Let $\omega_1$, $\omega_2$, and $\omega_3$ be three circles
	whose centers are not collinear. Show that the radical axes
	of each pair of circles concur at some point $P$,
	which is called the \vocab{radical center} of the three centers.
\end{problem}

\begin{problem}
	Describe a ruler-and-compass construction for the radical axis
	of \emph{any} two (possibly disjoint) circles.
\end{problem}

\begin{problem}[2019 AIME II/11]
	Triangle $ABC$ has side lengths $AB=7, BC=8,$ and $CA=9.$
	Circle $\omega_1$ passes through $B$ and is tangent to line $AC$ at $A.$
	Circle $\omega_2$ passes through $C$ and is tangent to line $AB$ at $A.$
	Let $K$ be the intersection of circles $\omega_1$ and $\omega_2$
	not equal to $A.$ Compute $AK$.
\end{problem}

\begin{niceprob}[Shortlist 2022 G2]
	In the acute-angled triangle $ABC$, the point $F$ is the foot of the altitude
	from $A$, and $P$ is a point on the segment $AF$. The lines through $P$ parallel
	to $AC$ and $AB$ meet $BC$ at $D$ and $E$, respectively. Points $X \ne A$ and $Y
	\ne A$ lie on the circles $ABD$ and $ACE$, respectively,
	such that $DA = DX$ and $EA = EY$.
	Prove that $B$, $C$, $X$, and $Y$ are concyclic.
\end{niceprob}

\begin{problem}[IMO 2000/1]
	Two circles $ G_1$ and $ G_2$ intersect at two points $ M$ and $ N$.
	Let $ AB$ be the line tangent to these circles at $ A$ and $ B$,
	respectively, so that $ M$ lies closer to $ AB$ than $ N$.
	Let $ CD$ be the line parallel to $ AB$ and passing through the point $ M$,
	with $ C$ on $ G_1$ and $ D$ on $ G_2$. Lines $ AC$ and $ BD$ meet at $ E$;
	lines $ AN$ and $ CD$ meet at $ P$; lines $ BN$ and $ CD$ meet at $ Q$.
	Show that $ EP = EQ$.
\end{problem}

\begin{niceprob}[Japan 2014/4]
	Let $\Gamma$ be the circumcircle of triangle $ABC$,
	and let $\ell$ be the tangent line of $\Gamma$ passing through $A$.
	Let $D$, $E$ be points on sides $AB$ and $AC$ such that $BD : DA = AE : EC$.
	Line $DE$ meets $\Gamma$ at points $F$ and $G$.
	The line parallel to $AC$ through $D$ meets $\ell$ at $H$,
	the line parallel to $AB$ through $E$ meets $\ell$ at $I$.
	Prove that there exists a circle passing through four points $F$, $G$, $H$, $I$,
	and tangent to line $BC$.
\end{niceprob}

\pagebreak

\section{Analytic Techniques}
\subsection{Theorems}
The following are things that we saw last year:
\begin{itemize}
	\ii (Extended) law of sines: In triangle $\triangle ABC$,
	\[2R = \frac{a}{\sin A} = \frac{b}{\sin B} = \frac{c}{\sin C}.\]
	\ii Law of cosines: $a^2 +  b^2 - 2ab\cos C = c^2$.
	\ii Ptolemy's inequality:
	\[AB\cdot CD + AD\cdot BC \ge AC\cdot BD,\]
	with equality when $ABCD$ is cyclic, or when
	$A$, $B$, $C$, $D$ are collinear, with the order of the points on the line
	being one of $(A,B,C,D)$, $(B,C,D,A)$, $(C,D,A,B)$, $(D,A,B,C)$.
	(Ptolemy's theorem is just the part where $ABCD$ is cyclic.)
	\ii Triangle inequality: $AB + BC \ge AC$, with equality if and only if
	$A$, $B$, $C$ are collinear in that order.
	\ii Stewart's theorem: $dad + man = bmb + cnc$.
	\ii Area formulas:
	$K = \frac{abc}{4R} = \sqrt{s(s - a)(s - b)(s - c)} = rs = ab\sin C$
\end{itemize}

You should also already know the trig angle addition formulas
from Algebra II or Precalc:
\begin{align*}
	\cos(x + y) &= \cos x\cos y - \sin x\sin y\\
	\sin(x + y) &= \sin x\cos y + \cos x \sin y.
\end{align*}
If you ever forget these, you can equate the real and imaginary parts of
\[e^{i(x + y)} = e^{ix} \cdot e^{iy}\]
by using $e^{i\theta} = \cos \theta + i\sin \theta$.

In addition, there are the product-to-sum and sum-to-product formulas below,
but I can never remember them. You should at least know how to derive them;
I rederive them every single time.
\begin{align*}
	\sin x\sin y &= \half(\cos(x - y) - \cos(x + y))\\
	\sin x\cos y &= \half(\sin(x - y) + \sin(x + y))\\
	\cos x\cos y &= \half(\cos(x - y) + \cos(x + y))\\
	\sin a + \sin b &= 2\sin\left(\frac{a + b}2\right)
	\cos\left(\frac{a - b}2\right)\\
	\cos a + \cos b &= 2\cos\left(\frac{a + b}2\right)
	\cos\left(\frac{a - b}2\right)
\end{align*}
The latter two equations are, of course, just the second and third equations
in the other direction
(i.e. $(x, y) = \left(\frac{a + b}{2}, \frac{a - b}{2}\right)$).

If you're not so familiar with any of these formulas,
then go through and verify them.

\pagebreak
\subsection{Walkthroughs}
\begin{example}[2015 AIME II/15]
	Circles $\mathcal{P}$ and $\mathcal{Q}$ have radii $1$ and $4$,
	respectively, and are externally tangent at point $A$.
	Point $B$ is on $\mathcal{P}$ and point $C$ is on $\mathcal{Q}$ so that
	line $BC$ is a common external tangent of the two circles.
	A line $\ell$ through $A$ intersects $\mathcal{P}$ again at $D$ and
	intersects $\mathcal{Q}$ again at $E$.
	Points $B$ and $C$ lie on the same side of $\ell$.
	Given that the areas of $\triangle DBA$ and $\triangle ACE$ are equal,
	determine their common value.
\end{example}
\begin{walkthrough}
	The key observation is that these circles are dilations of one another
	about $A$; we can leverage this fact to get a lot of length relations.
	\begin{enumerate}[label=\textbf{(\alph*)}]
		\ii Let any line through $A$ intersect $\mathcal P$ at $X$
		and $\mathcal Q$ at $Y$. Show that $\frac{AX}{AY} = \frac14$.
		\ii Show that $\angle BAC = 90\dg$, and then compute $AB$ and $AC$.
		One approach is to extend $AB$ to hit $\mathcal Q$ at $B'$
		and find similar triangles.
		\ii Let $\angle BAD = \alpha$. What are $\sin\alpha$ and $\cos\alpha$?
		(Hint: What do the equal areas tell you?)
		\ii What are $BD$ and $AD$?
		\ii If you did the previous parts correctly, you now know
		$AB$, $BD$, and $AD$; pick an area formula and finish.
	\end{enumerate}
\end{walkthrough}

\begin{example}[Fermat point]
	Let $\triangle ABC$ be a triangle whose angles are all less than $120\dg$
	and let $P$ be a point in the plane.
	Show that $PA + PB + PC$ achieves its minimum value if and only if
	$\angle APB = \angle BPC = \angle CPA = 120\dg$.
\end{example}

\begin{walkthrough}
	The trick with this problem is to draw $X$, the point on the other side
	of $BC$ from $A$ such that $\triangle BXC$ is equilateral.
	\begin{enumerate}[label=\textbf{(\alph*)}]
		\ii Let $F$ be the point such that
		$\angle AFB = \angle BFC = \angle CFA$. Show that $FBXC$ is cyclic and
		$A$, $F$, $X$ are collinear.
		Which two inequalities does $F$ therefore optimize?
		\ii Use the inequalities from the previous part
		to show that
		\[PA + PB + PC \ge AX,\]
		with equality if and only if $P = F$.
		\ii Where was the stipulation that $\angle A$, $\angle B$, and $\angle C$
		are less than $120\dg$ used?
		\ii Some extra things: If $Y$ and $Z$ are constructed similarly to $X$
		so that $\triangle CYA$ and $\triangle AZB$ are equilateral,
		show that $AX = BY = CZ$ and that those three lines are concurrent
		(at $F$).
	\end{enumerate}
	
	This is another instance where trying to understand the weird condition
	(namely, the equality case of $\angle AFB = \angle BFC = \angle CFA$)
	will get you pretty close to a solution;
	I found it by trying to construct $F$.
\end{walkthrough}

\begin{example}[2023 AIME I/12]
	Let $\triangle ABC$ be an equilateral triangle with side length $55$.
	Points $D$, $E$, and $F$ lie on sides $\overline{BC}$, $\overline{CA}$,
	and $\overline{AB}$, respectively, with $BD=7$, $CE=30$, and $AF=40$.
	A unique point $P$ inside $\triangle ABC$ has the property that
	\[\measuredangle AEP=\measuredangle BFP=\measuredangle CDP.\]
	Find $\tan\left(\measuredangle AEP\right)$.
\end{example}
\begin{walkthrough}
	I didn't solve this in contest, unfortunately, but here's a
	synthetic solution I found afterwards using the Fermat point.
	\begin{enumerate}[label=\textbf{(\alph*)}]
		\ii Show that the angle condition implies that
		$AEPF$, $BFPD$, and $CDPE$ are cyclic.
		What does the equilateral condition say about
		$\angle DPE$, $\angle EPF$, and $\angle FPD$?
		\ii Extend $DP$ to meet $(FPE)$ at $X$, so that it suffices to compute
		$\tan(\angle XDC)$. What do you know about $\triangle EFX$?
		What about quadrilateral $AXDC$?
		\ii Compute the length of $AX$ using Ptolemy's theorem.
		\ii Redraw quadrilateral $AXDC$ separately, and eradicate the problem.
	\end{enumerate}
	
	You can also approach this with complex numbers, coordinates, trigonometry,
	or other synthetic approaches
	(one I saw on AoPS was to drop the altitudes from $P$);
	I encourage looking for an alternate solution.
\end{walkthrough}

Finally, here's a number theory problem from IMO 2001
with a strange geometric solution.
\begin{example}[IMO 2001/6]
	Let $a > b > c > d$ be positive integers
	and suppose that
	\[ac + bd = (b + d + a - c)(b + d - a + c).\]
	Prove that $ab + cd$ is not prime.
\end{example}
\begin{walkthrough}
	This problem is\dots weird.
	Let's first reveal where the geometry really is:
	\begin{walk}
		\ii Expand the equation (yes, really). You should end up with
		\[a^2 - ac + c^2 = b^2 + bd + d^2.\]
		What do the left- and right-hand sides of this equation look like?
	\end{walk}
	In light of this, we can construct a quadrilateral $WXYZ$
	with sides $WX = a$, $XY = c$, $YZ = b$, $ZX = d$,
	and diagonal $WY$ having length
	\[\sqrt{a^2 - ac + c^2} = \sqrt{b^2 + bd + d^2},\]
	so that, by law of cosines, $\angle WXY = 60\dg$ and $\angle WZY = 120\dg$.
	This means $WXYZ$ is cyclic!
	
	On the other hand, the diagonals of a cyclic quadrilateral
	can \emph{always} be expressed in terms of their sides.
	\begin{walk}[resume]
		\ii Show that there are exactly three quadrilaterals
		with side lengths $a$, $b$, $c$, $d$ in some order.
		Moreover, show that, among these quadrilaterals,
		there are three possible lengths $t$, $u$, $v$ for the diagonals;
		that is, one quadrilateral has diagonals $(t, u)$,
		one has diagonals $(u, v)$, and one has diagonals $(v, t)$.
		
		\ii Apply Ptolemy's theorem in all three quadrilaterals
		to get the system of equations
		\begin{align*}
			ab + cd &= tu\\
			ac + bd &= uv\\
			ad + bc &= vt
		\end{align*}
		where $t = WY$ and $u = XZ$.
		Then, compute $t^2$.
		
		\ii Finally, we can use the inequality condition:
		show (ex. by rearrangement) that
		\[ab + cd > ac + bd > ad + bc,\]
		so, if $ab + cd$ is prime, then $t^2$ can never be an integer.
		However, from earlier, $WY^2 = b^2 + bd + d^2$ is an integer ---
		contradiction!
	\end{walk}
	
	The takeaway from here is primarily that quadratic expressions
	in two variables can sometimes be thought of as
	applications of the law of cosines.
	The additional geometric structure then lets you, well, do geometry
	to get nontrivial, non-geometric information
	(such as the fact that $(ac + bd)\mid (ab + cd)(ad + bc)$ in this case).
\end{walkthrough}

\pagebreak
\subsection{Extra Problems}

% requires trig ceva :facepalm:
%\begin{problem}[Shortlist 2001 G1]
%    Let $A_1$ be the center of the square inscribed in acute triangle $ABC$
%    with two vertices of the square on side $BC$.
%    Thus one of teh two remaining vertices of the square is on side $AB$
%    and the other is on $AC$.
%    Points $B_1$, $C_1$ are defined in a similar way for inscribed squares
%    with two vertices on sides $AC$ and $AB$, respectively.
%    Prove that lines $AA_1$, $BB_1$, $CC_1$ are concurrent.
%\end{problem}

\begin{problem}[NIMO, Evan Chen]
	Let $AXYZB$ be a convem pentagon inscribed in a semicircle with diameter $AB$.
	Suppose that $AZ - AX = 6$, $BX - BZ = 9$, and $BY = 5$.
	Find the perimeter of quadrilateral $OXYZ$, where $O$ is the midpoint of $AB$.
\end{problem}

\begin{problem}[2014 AIME I/15]
	In $\triangle ABC$, $AB = 3$, $BC = 4$, and $CA = 5$.
	Circle $\omega$ intersects $\overline{AB}$ at $E$ and $B$,
	$\overline{BC}$ at $B$ and $D$, and $\overline{AC}$ at $F$ and $G$.
	Given that $EF = DF$ and $\frac{DG}{EG} = \frac{3}{4}$, compute $DE$.
\end{problem}

\begin{problem}[2012 AIME I/13]
	Three concentric circles have radii 3, 4, and 5.
	An equilateral triangle with one vertex on each circle has side length $s$.
	Compute the largest possible area of the triangle.
\end{problem}

\begin{niceprob}[Pompeiu's theorem]
	Let $P$ be a point in the plane of equilateral triangle $\triangle ABC$.
	Show that $PA$, $PB$, and $PC$ form the sides of a (maybe degenerate) triangle.
	When is the triangle degenerate?
\end{niceprob}

\begin{problem}[AMC 12A 2021/24]
	Semicircle $\Gamma$ has diameter $\ol{AB}$ of length 14.
	Circle $\Omega$ lies tangent to $\ol{AB}$ at a point $P$
	and intersects $\Gamma$ at points $Q$ and $R$.
	If $QR = 3\sqrt3$ and $\angle QPR = 60\dg$, find the area of $\triangle PQR$.
\end{problem}

\begin{problem}[ARML 2023 T10]
	Parallelogram $ABCD$ is rotated about $A$ in the plane, 
	resulting in $AB'C'D'$, with $D$ on $\ol{AB'}$.
	Suppose that $[B'CD]=[ABD']=[BCC']$. Compute $\tan\angle ABD$.
\end{problem}

\begin{niceprob}[USAMO 1998/6]
	Let $n \geq 5$ be an integer. Find the largest integer $k$ (as a function of $n$)
	such that there exists a convex $n$-gon $A_{1}A_{2}\dots A_{n}$
	for which exactly $k$ of the quadrilaterals $A_{i}A_{i+1}A_{i+2}A_{i+3}$ have an
	inscribed circle, where indices are taken modulo $n$. 
\end{niceprob}

\end{document}