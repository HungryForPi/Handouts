\documentclass{scrartcl}
\usepackage{von}
\usepackage[usenames,dvipsnames,svgnames]{xcolor}
\usepackage[shortlabels]{enumitem}
\usepackage[framemethod=TikZ]{mdframed}
\usepackage{amsmath,amssymb,amsthm}
\usepackage{epigraph}
\usepackage[colorlinks]{hyperref}
\usepackage{microtype}
\usepackage{mathtools}
\usepackage[headsepline]{scrlayer-scrpage}
\usepackage{thmtools}
\renewcommand{\epigraphsize}{\scriptsize}
\renewcommand{\epigraphwidth}{60ex}
\addtolength{\textheight}{3.14cm}
\ihead{\footnotesize\textbf{Junior Math Team Notes}}
\ohead{\footnotesize \today}
\providecommand{\clubs}[1]{$#1\clubsuit$}
\providecommand{\ol}{\overline}
\providecommand{\eps}{\varepsilon}
\providecommand{\half}{\frac{1}{2}}
\providecommand{\dang}{\measuredangle} %% Directed angle
\providecommand{\CC}{\mathbb C}
\providecommand{\FF}{\mathbb F}
\providecommand{\NN}{\mathbb N}
\providecommand{\QQ}{\mathbb Q}
\providecommand{\RR}{\mathbb R}
\providecommand{\ZZ}{\mathbb Z}
\providecommand{\dg}{^\circ}
\providecommand{\ii}{\item}
\providecommand{\alert}[1]{{\sffamily\textbf{\textcolor{blue}{#1}}}}
% hacks for arc
\providecommand{\tarc}{\mbox{\large$\frown$}}
\providecommand{\arc}[1]{\stackrel{\tarc}{#1}}
\reversemarginpar
\providecommand{\printpuid}[1]{\marginpar{\href{https://otis.evanchen.cc/arch/#1}{\ttfamily\footnotesize\color{green!40!black}#1}}}

\mdfdefinestyle{mdbluebox}{roundcorner=10pt,innerbottommargin=9pt,
	linecolor=blue,backgroundcolor=TealBlue!5,}
\declaretheoremstyle[headfont=\sffamily\bfseries\color{MidnightBlue},
mdframed={style=mdbluebox},]{thmbluebox}

\mdfdefinestyle{mdredbox}{frametitlefont=\bfseries,innerbottommargin=8pt,
	nobreak=true,backgroundcolor=Salmon!5,linecolor=RawSienna,}
\declaretheoremstyle[headfont=\bfseries\color{RawSienna},
mdframed={style=mdredbox},headpunct={\\[3pt]},postheadspace=0pt,]{thmredbox}

\mdfdefinestyle{mdgreenbox}{linecolor=ForestGreen,backgroundcolor=ForestGreen!5,
	linewidth=2pt,rightline=false,leftline=true,topline=false,bottomline=false,}
\declaretheoremstyle[headfont=\bfseries\sffamily\color{ForestGreen!70!black},
mdframed={style=mdgreenbox},headpunct={ --- },]{thmgreenbox}

\mdfdefinestyle{mdorangebox}{linecolor=RedOrange,backgroundcolor=RedOrange!5,
	linewidth=2pt,rightline=false,leftline=true,topline=false,bottomline=false,}
\declaretheoremstyle[headfont=\bfseries\sffamily\color{RedOrange!70!black},
mdframed={style=mdorangebox},headpunct={ --- },]{thmorangebox}

\mdfdefinestyle{mdblackbox}{linecolor=black,backgroundcolor=RedViolet!5!gray!5,
	linewidth=3pt,rightline=false,leftline=true,topline=false,bottomline=false,}
\declaretheoremstyle[mdframed={style=mdblackbox}]{thmblackbox}

\declaretheorem[style=thmredbox,name=Problem,numberwithin=section]{problem}
\declaretheorem[style=thmredbox,name=Required Problem,sibling=problem]{reqproblem}
\declaretheorem[style=thmbluebox,name=Theorem,sibling=problem]{theorem}
\declaretheorem[style=thmbluebox,name=Lemma,sibling=theorem]{lemma}
\declaretheorem[style=thmbluebox,name=Theorem,numbered=no]{theorem*}
\declaretheorem[style=thmbluebox,name=Lemma,numbered=no]{lemma*}
\declaretheorem[style=thmgreenbox,name=Claim,sibling=theorem]{claim}
\declaretheorem[style=thmgreenbox,name=Claim,numbered=no]{claim*}
\declaretheorem[style=thmblackbox,name=Remark,sibling=theorem]{remark}
\declaretheorem[style=thmblackbox,name=Remark,numbered=no]{remark*}
\declaretheorem[style=thmgreenbox,name=Definition,sibling=theorem]{definition}
\declaretheorem[style=thmgreenbox,name=Definition,numbered=no]{definition*}
\declaretheorem[style=thmorangebox,name=Warning,sibling=theorem]{warning}
\declaretheorem[style=thmorangebox,name=Note,numbered=no]{note}
\declaretheorem[style=thmblackbox,name=Exercise,sibling=problem]{exercise}
%% 426c616e6b204c615465587e

\usepackage{graphicx}

\newenvironment{soln}{\begin{proof}[Solution]}{\end{proof}}

\newcommand{\vonsol}[1]{
	\begin{soln}
		\voninclude[1]{#1}
	\end{soln}
}

\title{Junior Math Team Inequalities Notes}
\subtitle{Based on lessons by Mr. Kats}
\author{Aditya Pahuja}
\date{\today}
\begin{document}
\maketitle
\tableofcontents
\pagebreak

\section{VB1}
The study of inequalities can largely be thought of as manipulations of
the following poorly-named inequality:
\begin{theorem}[Trivial Inequality, or VB1]
	For all real numbers $x$,
	\[x^2 \ge 0\]
	with equality if and only if $x = 0$.
\end{theorem}

Since the name ``trivial inequality'' seems a bit disparaging,
we instead refer to this as Very Basic 1, or VB1, as Mr. Kats
says\footnote{No, there is no VB2.}.
The main point to emphasize in this short section is that
this simple inequality is the basis for a large portion of the
theory of inequalities; it can be thought of as the ``machine code'' in which
the language of inequalities is run.
Therefore, when you use inequalities like AM-GM or Cauchy-Schwarz later on,
just know that you could in theory (and with a lot of pain)
distill these down to applications of good old VB1.

\begin{problem}
	Prove the AM-GM inequality for two variables; that is, prove that
	\[\frac{a + b}{2}\ge \sqrt{ab}\]
	for all positive real numbers $a$ and $b$,
	and show that equality occurs if and only if $a = b$.
\end{problem}

\begin{problem}[SophFrosh Practice]
	Compute the minimum value of the expression
	\[x^4y^2 + x^4 + x^2y^2 - 6x^2y + x^2 + y^2 + 1.\]
\end{problem}

\begin{problem}
	Let $a$, $b$, $c$, $d$, and $e$ be real numbers. Show that, if
	$2a < 5b$, then
	\[x^5 + ax^4 + bx^3 + cx^2 + dx + e\]
	cannot have all real roots.
\end{problem}

\section{Inequalities of Averages}
Probably the most well-known inequality, besides VB1, is the
\alert{arithmetic mean-geometric mean (AM-GM)} inequality,
which states that the arithmetic mean of any nonnegative reals
is at least as large as their geometric mean, with equality when
the numbers are all equal. Symbolically, it says
\begin{theorem}[AM-GM]
	If $a_1$, $a_2$, \dots, $a_n$ are nonnegative real numbers, then
	\[\frac{a_1 + a_2 + \cdots + a_n}{n} \ge \sqrt[n]{a_1a_2\cdots a_n},\]
	with equality if and only if $a_1 = a_2 = \cdots = a_n$.
\end{theorem}

The previous section proposed two-variable AM-GM as a practice problem;
for completeness, we now prove it here:
\begin{align*}
	\frac{a+b}{2} &\ge \sqrt{ab}\\
	\iff \left(\frac{a+b}{2}\right)^2 &\ge ab\\
	\iff a^2 + 2ab + b^2 &\ge 4ab\\
	\iff a^2 - 2ab + b^2 = (a - b)^2 &\ge 0.
\end{align*}
Observe that each manipulation made here is reversible,
so we've constructed a chain of equivalent inequalities,
the last of which is true; this implies that the original inequality is
also true.
Equality of course holds only when $a - b = 0$, or equivalently $a = b$.

Using the two-variable version, we can easily prove Theorem 2.1
whenever $n = 2^k$ is a power of two (try it yourself!).
One way to do this is by induction on $k$.

\begin{theorem*}[AM-GM on $2^k$ variables]
	Let $n = 2^k$ for some positive integer $k$. Then,
	\[\frac{a_1 + a_2 + \cdots + a_n}{n} \ge \sqrt[n]{a_1a_2\cdots a_n},\]
	and equality holds if and only if $a_1 = a_2 = \cdots = a_n$.
\end{theorem*}
\begin{proof}
	We perform induction on $k$, with the base case $k = 1$ being proven above. 
	
	For the inductive hypothesis, assume that this holds
	up to some arbitrary $k$.
	We then wish to show that the inequality is also true for $k + 1$:
	\begin{align*}
		\frac{a_1 + a_2 + a_3 + \cdots + a_{2^{k+1}}}{2^{k+1}} &=
		\half\left(\frac{a_1 + a_2 + \cdots + a_{2_k}}{2^k}
		+ \frac{a_{2^k+1} + a_{2^k+2} + \cdots + a_{2^{k+1}}}{2^k}\right)\\
		&\ge \half\left(\sqrt[2^k]{a_1a_2\cdots a_{2^k}} +
		\sqrt[2^k]{a_{2^k+1}a_{2^k+1}\cdots a_{2^{k+1}}}\right)\\
		&\ge \sqrt{\sqrt[2^k]{a_{2^k+1}a_{2^k+1}\cdots a_{2^{k+1}}}\cdot
		\sqrt[2^k]{a_1a_2\cdots a_{2^k}}}\\
		&= \sqrt[2^k]{a_{1}a_{2}a_3\cdots a_{2^{k+1}}}.
	\end{align*}
	Intuitively, we're halving the dataset into two smaller powers of two
	on which we can apply our smaller AM-GMs.
\end{proof}

This extends readily into all positive integers with the following trick:
\begin{problem}
	Show that if the AM-GM inequality is true for $n \ge 3$ variables,
	then it must also be true for $n - 1$ variables.
	(Be careful not to use circular reasoning here!)
\end{problem}
In principle, the idea is that we can find arbitrarily large $n$
for which AM-GM is true, so the ability to prove $n\implies n - 1$
covers all of $\NN$. 

\end{document}